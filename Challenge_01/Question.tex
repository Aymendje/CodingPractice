\documentclass[12pt]{article}

\usepackage{geometry}
\geometry{letterpaper,tmargin=1in,bmargin=1in,lmargin=1in,rmargin=1in}

\usepackage[utf8]{inputenc}
\usepackage[english]{babel}
\usepackage{listings}

\setlength\parskip{0.1in}
\newcommand{\code}[1]{\texttt{#1}}
\newcommand{\dcode}[1]{\[\texttt{#1}\]}

\usepackage[final]{pdfpages}
\usepackage{fullpage}

\makeatletter
\newenvironment{chapquote}[2][2em]
  {\setlength{\@tempdima}{#1}%
   \def\chapquote@author{#2}%
   \parshape 1 \@tempdima \dimexpr\textwidth-2\@tempdima\relax%
   \itshape}
  {\par\normalfont\hfill--\ \chapquote@author\hspace*{\@tempdima}\par\bigskip}
\makeatother

\begin{document} 
\lstset{language=python}

\section*{\hfil--- Problem 1: Santa's ladder ---\hfil}

 	\begin{chapquote}{Brooke Hampton, \textit{Writer for the Huffington Post}}
 	``Maybe we should spend less time teaching kids to believe in Santa and more time teaching them to believe in themselves.''
 	\end{chapquote}


    Santa is trying to deliver presents in a large apartment building, but he can't find the right floor - the directions he got are a little confusing. He starts on the ground floor (floor 0) and then follows the instructions one character at a time.


	\subsection* {Where is Santa}

    An opening parenthesis, (, means he should go up one floor, and a closing parenthesis, ), means he should go down one floor.
    
    The apartment building is very tall, and the basement is very deep; he will never find the top or bottom floors.
    
    For example:
    
    \begin{lstlisting}
        (()) and ()() both result in floor 0.
        ((( and (()(()( both result in floor 3.
        ))((((( also results in floor 3.
        ()) and ))( both result in floor -1 (the first basement level).
        ))) and )())()) both result in floor -3.
    \end{lstlisting}
    
    \textbf{To what floor do the instructions take Santa?}  

	\subsection* {The basement}

    Now, given the same instructions, find the position of the first character that causes him to enter the basement (floor -1). The first character in the instructions has position 1, the second character has position 2, and so on.
    
    For example:
    
    \begin{lstlisting}
        ) causes him to enter the basement at character position 1.
        ()()) causes him to enter the basement at character position 5. 
    \end{lstlisting}
    
    \textbf{What is the position of the character that causes Santa to first enter the basement?}  
    
\end{document}
