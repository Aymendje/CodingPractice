\documentclass[12pt]{article}

\usepackage{geometry}
\geometry{letterpaper,tmargin=1in,bmargin=1in,lmargin=1in,rmargin=1in}

\usepackage[utf8]{inputenc}
\usepackage[french]{babel}
\usepackage{listings}

\setlength\parskip{0.1in}
\newcommand{\code}[1]{\texttt{#1}}
\newcommand{\dcode}[1]{\[\texttt{#1}\]}

\usepackage[final]{pdfpages}
\usepackage{fullpage}

\usepackage{lipsum}
\newcounter{examplecounter}
\newenvironment{example}{\begin{quote}%
}{%
\end{quote}%
}

\makeatletter
\newenvironment{chapquote}[2][2em]
  {\setlength{\@tempdima}{#1}%
   \def\chapquote@author{#2}%
   \parshape 1 \@tempdima \dimexpr\textwidth-2\@tempdima\relax%
   \itshape}
  {\par\normalfont\hfill--\ \chapquote@author\hspace*{\@tempdima}\par\bigskip}
\makeatother

\begin{document}
\lstset{language=python}

\section*{\hfil--- Problème 1: L'ascenseur Noël ---\hfil}

  \begin{chapquote}{Brooke Hampton, \textit{Chroniqueur au Huffington Post}}
  ``Nous devrions peut-être passer moins de temps à enseigner aux enfants à croire au Père Noël et plus de temps à leur apprendre à croire en eux même.''
  \end{chapquote}


   Le Père Noël cherche à offrir des cadeaux dans un grand immeuble, mais il ne peut pas trouver le bon étage - les directions qu'il a obtenu sont un peu confuses. Il commence au rez-de-chaussée (étage 0), puis suit les instructions un caractère à la fois.


  \subsection* {Où est le père noël}

    Une parenthèse ouvrante, (, signifie qu'il devrait monter un étage, et une parenthèse fermante, ), signifie qu'il devrait descendre d'un étage.

    L'immeuble est très grand, et le sous-sol est très profond; il ne pourra jamais atteindre l'étage le plus haut, ni le plus bas.

    Par exemple:
    
    \begin{example}
      (())  et  ()()    réultent tout les deux à l'étage 0.\\
      (((   et  (()(()( réultent tout les deux à l'étage 3.\\
      ))(((((           fini également au 3ieme étage.\\
      ())   et  ))(     les deux résultat à l'étage -1 (le premier niveau du sous-sol).\\
      )))   et  )())()) sont tout les deux à l'étage -3.
    \end{example}
    
    \textbf{Après avoir suivis les instructions, a quel étage le Père Noël finiras t'il ?}  

  \subsection* {Le sous-sol}

    Maintenant, étant donné les mêmes instructions, trouver la position du premier caractère qui l'amène à entrer dans le sous-sol (niveau -1). Le premier caractère dans les instructions a la position 1, le deuxième personnage a la position 2, et ainsi de suite.

    Par exemple:
    
    \begin{example}
      )     Le fait entrer dans le sous-sol à la position de caractère 1.\\
      ()())   Le fait entrer dans le sous-sol à la position de caractère 5.
    \end{example}
    
    \textbf{Quelle est la position du caractère qui provoque le Père Noël pour la première entrer dans le sous-sol?}  
    
\end{document}
